Sliding window approaches make use of feature pyramids to detect objects at different scales and locations in an image. To compute a  feature pyramid, first an image pyramid is constructed. Image pyramids are constructed via repeated subsampling of the original image. The subsampling factor between any two consecutive pyramid levels is constant and determined via a parameter $\lambda$. This parameter $\lambda$ defines the number of levels in an octave. An octave in turn is defined as the number of levels that have to be stepped down in a pyramid until an image of twice the resolution is reached. Figure \ref{fig:impyra} visualises an image pyramid with two octaves and $\lambda=1$. The bottom level of an image pyramid is usually at twice the resolution of the original image. In case an extra-octave pyramid or CNN features are used, the original image is resized to twice the resolution before the construction of the image pyramid, which results in a bottom level that has four times the resolution of the original image. Note that only the parts will be placed in the higher resolution bottom octave of the pyramid. After the image pyramid is constructed, feature maps are computed for every pyramid level which results in the desired feature pyramid (see figure \ref{fig:featpyra}).
During this project a 3D Deformable Parts Model was implemented and carefully evaluated with respect to different design choices and model parameters. The result of this work is a 3D model whose performance on object detection and pose-estimation is on par and in many cases even better than results reported for other 3D-DPMs \cite{Pepik:2012aa} \cite{6248075}. Through the evaluation of several design choices and model parameters, deeper insights into the training of 3D-DPMs and DPMs in general have  been gained. Among the factors that turned out to be important, were the construction of the feature pyramid where an extra octave (to detect small objects) showed significant performance improvements. The way how truncated and occluded objects are handled turned out to be crucial as well, and it was found that performance improves significantly by the use of a truncation feature and effective handling of bounding-box overlap requirements during training. It has also been shown how different types of CAD examples affect the performance of the model. Here, perspective renderings on uniform backgrounds have proofed to work best. The introduction of a penalty term during latent positive search has turned out to be an effective way to increase the viewpoint accuracy of the 3D-DPM.

While the model has shown to perform well on viewpoint estimation, the estimations at the moment are still limited to a set of discrete viewpoints the model was trained on. While it is in theory possible to train the model on arbitrarily many discrete viewpoints (provided that sufficient CAD renderings are generated), this is clearly not an efficient solution. An idea for a new and alternative approach is outlined in section \ref{sec:future}.

The incorporation of features obtained by Convolutional Neural Networks has been found to be more difficult than expected. Feature standardisation, an adjustment of the deformation lower-bounds and a modification of the training procedure have proofed to be essential to get a comparable detection performance to HOG features. Also some compromise between performance and training time had to be made in order to make experiments feasible (reducing the number of pyramid levels mainly). While the detection performance of HOG features can nearly be matched by CNN based models using these modifications, the viewpoint estimation capabilities of CNN based models showed to be significantly worse. By using wrong viewpoint assignments as negative examples during training, I managed to increase the pose-estimation performance with CNN features such that they even exceed the performance of HOG based models on datasets where objects are not occluded or truncated. 

%There were several challenges to be mastered during the work on this project. The first (and probably the most overwhelming) being getting to know and understand the rather complex implementation of the DPM. This was crucial though as making modifications to the code very often affected several other portions of it, which had to be adjusted or modified.  I tried to refactor the code in several ways to make it more modular and flexible. The lesson learned here is, that code, which is well designed with respect to clearly stated and distributed responsibilities, greatly simplifies modifications. 
%
%Looking back, another lesson learned during the work on the project is that of verifying modifications after they have been applied. Too long I was working on the code, making several often massive changes, without controlling their effect on the model's performance. Of course I made sure that it still runs error free, but this does  This of course made finding and solving "bugs" unnecessarily more difficult than it would have been if rigorous tests would have been performed after each major modification. The time needed for training and evaluating a model on a PC however often discouraged such tests and I only really started with such rigorous testing when working on the cluster.  

\section{Future work}\label{sec:future}
One idea of future research work on the model  aims to increase the model's viewpoint accuracy, and to also allow for finer viewpoint estimation. Rather than using interpolation between filters at test-time to allow finer viewpoint estimation  as in \cite{Pepik:2012aa} (which increases detection time), the idea is to use a voting scheme to determine the viewpoint. Given an image of a car, there are typically going to be several overlapping object hypotheses with different viewpoint assignments and scores around the correct object localisation. Rather than just applying non-maximal suppression on all these hypotheses, the non-maximal suppression could only be applied to groups of possible detections with the same viewpoint-assignment only and the resulting set of hypotheses used to arrive at a better, more complete object hypothesis. The most confident (highest scoring) hypothesis  could  be used as an initial guess for the viewpoint and location of the object and hypotheses with neighbouring viewpoints as well as a certain overlap with the initial guess could then be used to refine the initial guess by voting for other viewpoints (with voting-weights  being proportional to the hypothesis score). This approach would also not lead to a significant increase in detection time as there are not more convolutions of the filters with the feature maps needed. 

Another path for future work could be the improvement of the CNN based models by fine-tuning the network weights to the given task of object detection with Deformable Parts Models. As such fine-tuning has shown to be very effective by \cite{girshick2013rich}, this could result in much better performance. To adapt the fine-tuning procedure to the intended use with DPMs, region proposals with a bounding box overlap of more than 0.5 as well as smaller regions totally inside the bounding box (related to possible parts) could be used as positive examples. 

It would also be interesting to test the models performance on other object classes, especially with CNN features as they have shown to improve over HOG features for other object classes than cars. This of course means that training data in the form of CAD examples would first have to be created. 